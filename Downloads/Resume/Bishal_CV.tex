%%%%%%%%%%%%%%%%%%%%%%%%%%%%%%%%%%%%%%%
% Hieu Do - Resume
% 7/26/2016
%
% Reference:
% Debarghya Das (http://debarghyadas.com)

\documentclass[]{hieudo-build}
%\usepackage{ragged2e}



\begin{document}

%%%%%%%%%%%%%%%%%%%%%%%%%%%%%%%%%%%%%%
%
%     TITLE NAME
%
%%%%%%%%%%%%%%%%%%%%%%%%%%%%%%%%%%%%%%
\namesection{Thingom Bishal Singha}
{\urlstyle{same}
	\faEnvelope \href{mailto:bishalthingom@gmail.com}{ bishalthingom@gmail.com}\\
	\faGithub \href{https://github.com/bishalthingom}{   github.com/bishalthingom}\\
	\faLinkedinSquare \href{https://www.linkedin.com/in/bishal-thingom/}{   linkedin.com/in/bishal-thingom}
}


    
%%%%%%%%%%%%%%%%%%%%%%%%%%%%%%%%%%%%%%
%
%     COLUMN ONE
%
%%%%%%%%%%%%%%%%%%%%%%%%%%%%%%%%%%%%%%
\begin{minipage}[t]{0.34\textwidth} 

%%%%%%%%%%%%%%%%%%%%%%%%%%%%%%%%%%%%%%
%     EDUCATION
%%%%%%%%%%%%%%%%%%%%%%%%%%%%%%%%%%%%%%
\section{Education} 

\subsection{National Institute of Technology \break Karnataka, Surathkal}
\timeplace{May 2016 | Mangalore, India\\}
\descript{B.Tech. in Computer Engineering}
\bold{GPA}: 7.17 / 10
\sectionsep

\subsection{Kendriya Vidyalaya, Hebbal}
\timeplace{May 2012 | Bangalore, India\\}
\descript{Senior School Certificate (CBSE)}
\bold{Score}: 94.6 \%
\sectionsep

%%%%%%%%%%%%%%%%%%%%%%%%%%%%%%%%%%%%%%
%     SKILLS
%%%%%%%%%%%%%%%%%%%%%%%%%%%%%%%%%%%%%%
\section{Skills}
% \subsection{Programming}
\emphasis{Languages:}
\regular{Python, C, C++, Java, C\#,\\Visual Basic, Powershell}
\vspace{2mm}
\emphasis{ML Libraries:}
\regular{Keras, Tensorflow, NLTK, \\Scikit-learn, Pandas}
\vspace{2mm}
\emphasis{Web Development:}
\regular{HTML5, CSS3, PHP }
\vspace{2mm}
\emphasis{Databases:}
\regular{BigQuery, SQL Server}


% \sectionsep
% \subsection{Languages}
% \location{Native fluency:} English, Vietnamese\\
\sectionsep

% %%%%%%%%%%%%%%%%%%%%%%%%%%%%%%%%%%%%%%
% %     HACKATHONS
% %%%%%%%%%%%%%%%%%%%%%%%%%%%%%%%%%%%%%%
% \section{Hackathons}
% HackMIT \textbullet{} hackNY \\
% WearHacks NY \textbullet{} Hackademics VN \\
% \sectionsep


%%%%%%%%%%%%%%%%%%%%%%%%%%%%%%%%%%%%%%
%     COURSEWORK
%%%%%%%%%%%%%%%%%%%%%%%%%%%%%%%%%%%%%%


%%%%%%%%%%%%%%%%%%%%%%%%%%%%%%%%%%%%%%
%     ADDITIONAL INFORMATION
%%%%%%%%%%%%%%%%%%%%%%%%%%%%%%%%%%%%%%


% %%%%%%%%%%%%%%%%%%%%%%%%%%%%%%%%%%%%%%
% %     AWARDS
% %%%%%%%%%%%%%%%%%%%%%%%%%%%%%%%%%%%%%%

% \section{Awards} 
% Dean's List\\
% NYU PROMISE Scholarship\\
% Shelby C. Davis Scholarship\\
% President’s Circle Scholarship\\
% \sectionsep

\sectionsep
\DTMsetdatestyle{mylastupdate}
\DTMdisplaydate{\the\year}{\the\month}{\the\day}{-1}

%%%%%%%%%%%%%%%%%%%%%%%%%%%%%%%%%%%%%%
%
%     COLUMN TWO
%
%%%%%%%%%%%%%%%%%%%%%%%%%%%%%%%%%%%%%%
\end{minipage} 
\hfill
\begin{minipage}[t]{0.65\textwidth} 

%%%%%%%%%%%%%%%%%%%%%%%%%%%%%%%%%%%%%%
%     EXPERIENCE
%%%%%%%%%%%%%%%%%%%%%%%%%%%%%%%%%%%%%%
\section{Experience}

\workplace{Visa Inc.}{Aug 2016 – present}\\
\position{Software Engineer}{Bangalore, India}
\vspace{0.9em} % Hacky fix for awkward extra vertical space
\begin{tightemize}
\item \justified{\regular{Led multiple Performance engg. projects for the Merchant and Acquirer Processing Platform, achieving a 100\% success rate in production.}}
\item \justified{\regular{Developed toolset to automate Environment validation and correction, reducing manual effort on each run from 2 working days to 2 hours.}}
\item \justified{\regular{Developed regression tool to generate automated reports on usage of system resources.}}

\end{tightemize}
\sectionsep

\workplace{JPMorgan Chase \& Co.}{May 2015 – July 2015} \\
\position{Technology Intern}{Bangalore, India}
% \vspace{\topsep} % Hacky fix for awkward extra vertical space
\begin{tightemize}
\item \justified{\regular{Built an audit preparation tool to generate a checklist of tasks and predict the next audit period using relational DBMS and linear regression.}} 
\end{tightemize}
\sectionsep

%%%%%%%%%%%%%%%%%%%%%%%%%%%%%%%%%%%%%%
%     PROJECTS
%%%%%%%%%%%%%%%%%%%%%%%%%%%%%%%%%%%%%%
\section{Projects}
\project{Person Recognition using Smartphones’ Accelerometer Data}{\href{https://arxiv.org/pdf/1711.04689}{arXiv}}
\extra{Under review at 24th IEEE National Conference on Communications}
\justified{\regular{Designed a model to recognize a person using Random Forests, following feature analysis in time and frequency domains.
Experimental results outperformed existing models with an accuracy of 96.79\% and AUC of 98.22\%.}}
\sectionsep

\project{BookSim for Multi-layered Networks}
{\href{https://github.com/bishalthingom/booksim2-layered}{GitHub}}\\
\justified{\regular{Modified the existing dimension ordered routing algorithm to accommodate multiple layers, and isolated the configurations for each dimension. The modified version was thus able to simulate multi-layered networks and successfully reflect the smaller latency for the inter-layer connections. }}
\sectionsep 

\project{Pokemon Type Classifiation using Transfer Learning}
{\href{https://github.com/bishalthingom/pokemon-transferlearning}{GitHub}}\\
\justified{\regular{Fine tuned the VGG16 Convolutional Neural Network to determine the type of a Pokemon, given its image. Using image augmentation and transfer learning on a limited dataset of 3500 images, the resultant model delivered an accuracy of 0.396. on an 18-class classification problem.  }}
\sectionsep 

\project{Similarity of Songs' Lyrics across Genres and Times}
{\href{https://bishalthingom.wordpress.com/portfolio/clustering-how-do-lyrics-vary-across-genres-and-time/}{Wordpress} | \href{https://github.com/bishalthingom/song-clustering}{GitHub}}\\
\justified{\regular{Tf-idf vectorization was applied on the song lyrics, and then the Euclidean distance between the vectors was calculated. Pairwise k-medoids clustering was applied on the resultant graph and the similarity of genres/decades was determined based on the clustering results.}}
\sectionsep 

\project{Circles of Parity using Hamiltonian Cycles}
{\href{https://bishalthingom.wordpress.com/portfolio/circle-of-parity/}{Wordpress} | \href{https://github.com/bishalthingom/circle-of-parity}{GitHub}}\\
\justified{\regular{Modified the cost-function of Held-
Karp algorithm to the longest edge in the cycle for extracting the earliest "circle of parity" (Hailtonian cycle) in a graph representing game results in a league season. The number of circles of parity was determined using edge-disjoint Hamiltonian cycles.}}
\sectionsep 

\project{Hawkeye Visualization using graphics.h}
{\href{https://github.com/bishalthingom/hawkeye-visualization-cpp}{GitHub}}\\
\justified{\regular{Visualized LBW in cricket using inputs from two planes and parabolic motion for ball trajectory. Depth visualization was achieved by varying measures in x and y dimensions as a linear function of measures in z dimension.}}
\sectionsep 

\end{minipage} 

\end{document}  
